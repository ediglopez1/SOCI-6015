\documentclass[11pt]{article} %tamño de letra, tipo de artículo
\usepackage[1.5]{setspace}    % Permite ajustar el espaciado entre líneas (simple, 1.5, doble).
\usepackage{graphicx}    % Facilita la inclusión y manipulación de imágenes en el documento.
%\usepackage{geometry}    
\usepackage{amsmath}     % Mejora el entorno matemático de LaTeX para expresiones complejas.
\usepackage{natbib}      % Gestiona citas y referencias bibliográficas con estilos variados.
\usepackage{array}       % Amplía las capacidades de formateo de tablas en el entorno array.
\usepackage{multirow}    % Permite crear celdas que abarquen varias filas en tablas.
\usepackage{siunitx}     % (Reemplaza a dcolumn) Alinea columnas numéricas y maneja unidades de manera avanzada.
\usepackage{textcomp}    % Proporciona símbolos adicionales como el símbolo del grado, número y el euro.
\usepackage[utf8]{inputenc}  % Configura la codificación del documento para usar UTF-8.
\usepackage[T1]{fontenc}     % Configura la salida del documento con codificación T1, adecuada para caracteres en español.
\usepackage[spanish]{babel}  % Configura el documento para usar el idioma español, ajustando reglas tipográficas y de separación de sílabas.
\usepackage{csquotes}        % Proporciona comillas tipográficas adecuadas para el español.
\usepackage[top=1in,right=1in,left=1in,bottom=1in]{geometry} % Ajusta los márgenes y la disposición de la página. Este indica que el margen es de una pulgada a cada lado.

\bibpunct{(}{)}{;}{a}{}{,}
\parindent 1.5em %\raggedright

\usepackage{hyperref}        % Convierte referencias cruzadas y URLs en hipervínculos dentro del PDF.

\hypersetup{
    unicode=true,            % Permitir caracteres no latinos en los marcadores.
    pdftoolbar=true,         % Mostrar la barra de herramientas de Acrobat.
    pdfmenubar=true,         % Mostrar el menú de Acrobat.
    pdffitwindow=false,      % Ajustar la ventana a la página al abrir.
    pdfstartview={FitH},     % Ajusta el ancho de la página a la ventana.
    pdftitle={},             % Título del documento.
    pdfauthor={},            % Autor del documento.
    pdfsubject={},           % Asunto del documento.
    pdfcreator={},           % Creador del documento.
    pdfproducer={},          % Productor del documento.
    pdfkeywords={},          % Lista de palabras clave.
    pdfnewwindow=true,       % Abrir enlaces en una nueva ventana.
    colorlinks=true,         % Enlaces en color (sin recuadro).
    linkcolor=blue,          % Color de los enlaces internos.
    citecolor=blue,          % Color de los enlaces a la bibliografía.
    filecolor=blue,          % Color de los enlaces a archivos.
    urlcolor=blue            % Color de los enlaces externos.
}
\title{SOCI 6015\\ Asignación \textnumero 1}

\author{Pon tu nombre aquí}

\date{\today}

\begin{document}

\maketitle
\begin{center}
   A entregarse el 27 de agosto de 2024. 
\end{center}


\section{Problemas matemáticos}

Para los siguientes problemas, resuelva mostrando los pasos intermedios.

\subsection{Problema 1 - 5 puntos} 

Resuelva la ecuación siguiente: $3x + 7 = 2x - 5$. Use \texttt{align} o \texttt{eqnarray}. Si tiene dudas, consulte \href{https://tex.stackexchange.com}{Stack Overflow}, o venga a mis horas de oficina.

\subsection{Problema 2 - 5 puntos} 

Resuelve la desigualdad: $2(3x - 4) > 5x + 8 - x$. 

\section{Temas de interés - 20 puntos}

Cree una tabla sencilla con dos columnas y tres filas. En la primera columna, escriba tres temas sociológicos que le interesen. En la segunda columna, explique brevemente por qué le interesan estos temas. Asegúrese de incluir un título para la tabla. 

Luego, inserte una imagen (al tamaño que usted mejor entienda en relación al ancho de página) que represente uno de los temas que mencionara en la tabla. Debajo de la imagen, incluya un pie de foto explicando por qué seleccionó esa imagen y cómo se relaciona con el tema. 

Finalmente, escriba un breve párrafo (unas 4 oraciones) al final del documento, \underline{después de la imagen}, donde explique cuál de los tres temas te llama más la atención para explorar a fondo durante el curso y/o maestría y por qué. Use \textbf{negrita}, \textit{cursiva}, o \underline{subrayado} en algún momento durante ese párrafo para enfatizar algo que entienda sea importante que yo entienda.

\end{document}





